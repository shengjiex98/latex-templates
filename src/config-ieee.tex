% For hyperlinks
\usepackage{hyperref}
% For SI units
\usepackage{siunitx}
% Inline enumeration
\usepackage[inline]{enumitem}
% Used to define acronyms.
\usepackage{glossaries}

% For formatting figures
\usepackage{graphicx}
% For loading Inkscape TeX+PDF figures
\usepackage{import}
% For beautiful tables.
\usepackage{booktabs}
% Multirow cells in tables
\usepackage{makecell, multirow}

% === IEEE only ===
% For improved citation styling
\usepackage{cite}
% For the ease of theorem setup.
\usepackage{amsmath,amsthm,amssymb}
% Define the "Problem" environment
\newtheorem{problem}{Problem}

% Use color in tables
\usepackage[prologue,table,svgnames]{xcolor}

% Workaround to make subcaption not take over captioning from the IEEEtran class
% Subcaption is used for subfigures
\makeatletter
\let\MYcaption\@makecaption
\makeatother
\usepackage{subcaption}
\makeatletter
\let\@makecaption\MYcaption
\makeatother
% <<<<<<<<

% Algorithm environment
\usepackage[ruled,lined,noend]{algorithm2e}
\SetKw{Continue}{continue}
% Line numbers
\usepackage{lineno}

% For normalizing how we reference figures, tables, and equations. It should
% be the last package to import to make sure all reference types are recognized.
\usepackage[capitalize,noabbrev]{cleveref}

% \documentclass[conference]{IEEEtran}
% \IEEEoverridecommandlockouts
